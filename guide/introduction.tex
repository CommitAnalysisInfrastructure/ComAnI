\section{Introduction}
\label{sec:Introduction}
The Commit Analysis Infrastructure (\thetool{}) is an open and configurable infrastructure for the extraction and analysis of commits from software repositories. For both tasks, individual plug-ins realize different extraction and analysis capabilities, which rely on the same data model provided by the infrastructure. Hence, any combination of extraction and analysis plug-ins is possible. For example, we could first conduct an analysis for a software hosted in a Git repository \cite{Git18} and later conduct the same analysis for a different software hosted by SVN \cite{Svn18}\footnote{Assuming that the analysis is able to cope with the artifacts and their technologies of the new software under analysis.}. Another example is to use the same commit extractor, e.g., supporting the commit extraction from Git repositories, for different analyses. The definition of a particular \thetool{} instance consists of a set of configuration parameters saved in a configuration file, which the infrastructure reads at start-up. Hence, there is no implementation effort needed. The infrastructure automatically performs its internal setup, loads and starts the desired plug-ins.

\thetool{} represents a large increment of the ComAn toolset~\cite{Com18a}. This toolset uses a single commit extraction script and a Java-based implementation of a particular commit analysis~\cite{KS17a, KS17b, KGS18}. Further, the toolset is designed to be applied to the Linux kernel~\cite{Lin18a} and its Git repository~\cite{Lin18b}. This design of ComAn restricts its applicability to other software and repository types. While it is not completely impossible to adapt it to other inputs, this adaptation requires mayor implementation effort. Hence, we decided to create a complete infrastructure, which realizes a flexible and highly configurable ecosystem for conducting a variety of analyses by means of plug-ins for commit extraction and their analysis.

This guide consists of three parts. The first part in Section~\ref{sec:Overview} introduces \thetool{} in more detail and describes the concepts realizing the core features of the infrastructure. Section~\ref{sec:UserGuide} represents the second part, which focuses on the end user of \thetool{}. We describe how to download, install and execute the core infrastructure as well as the available plug-ins. As part of the execution, we also discuss the configuration parameters and the definition of particular \thetool{} instances. The third part of this guide focuses on the developers. In Section~\ref{sec:DeveloperGuide}, we discuss the development of new extraction and analysis plug-ins by examples.
