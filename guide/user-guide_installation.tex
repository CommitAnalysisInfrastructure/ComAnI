\subsection{Installation}
\label{sec:Installation}
This section describes the installation of the main infrastructure as well as how to satisfy its requirements. We provide this description for both operating systems currently used to execute \thetool{}: Ubuntu and Windows. However, the first part in Section~\ref{sec:Download} is operating system independent, while Section~\ref{sec:Requirements} contains descriptions for both operating systems.

\subsubsection{Downloading \thetool{}}
\label{sec:Download}
We recommend downloading the \fselement{\thetool{.zip}} archive, which already contains the main infrastructure, the guide, and the plug-ins developed by the SSE group. This requires the following steps, which are operating system independent:
\begin{enumerate}
	\item Go to \url{https://github.com/CommitAnalysisInfrastructure/ComAnI}
	\item Navigate to the release directory
	\item Click on the \fselement{\thetool{.zip}} file
	\item Click the Download button and save the archive
	\item Go to the download directory and extract all files in the \thetool{.zip} archive to your favorite directory
\end{enumerate}
The directory to which the \thetool{-plug-ins} are extracted in step 5 above is used for the definition of the configuration file in Section~\ref{sec:Execution}.

\subsubsection{Satisfying Requirements}
\label{sec:Requirements}
The execution of \thetool{} requires the installation of Java 8 or higher (or equivalents, like OpenJDK). The required steps for this installation are operating system dependent Hence, we first describe the necessary steps for Ubuntu and then for Windows below.

\textbf{On Ubuntu}, the following steps have to be performed:
\begin{enumerate}
	\item Open a Terminal
	\item Update the apt package index: \clcommand{sudo apt update}
	\item Install OpenJDK:
	\begin{enumerate}
		\item Either the default, if it is OpenJDK 8 or higher: \clcommand{sudo apt install default-jdk}
		\item Or explicitly OpenJDK 8: \clcommand{sudo apt install openjdk-8-jdk}
	\end{enumerate}
	\item Check if the installation was successful by printing the installed Java version: \clcommand{java -version}
\end{enumerate}

\textbf{On Windows}, the following steps have to be performed:
\begin{enumerate}
	\item Go to \url{https://www.java.com/en/download/} (Oracle Java)
	\item Download the Java Runtime Environment (JRE) for your system\footnote{If you want to develop your own \thetool{-plug-ins}, you need to download and install the Java SE Development Kit (JDK), e.g., here \url{https://www.oracle.com/technetwork/java/javase/downloads/jdk8-downloads-2133151.html}}
	\item Execute the installer and follow its instructions
	\item Check if the installation was successful by printing the installed Java version:
	\begin{enumerate}
		\item Open the Command Prompt: Hit the Windows-Key or click Start, type \clcommand{cmd} and hit Enter
		\item In the Command Prompt: \clcommand{java -version}
	\end{enumerate}
\end{enumerate}